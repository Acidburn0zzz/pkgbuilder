% !TEX TS-program = xelatex
% !TEX encoding = UTF-8 Unicode

% Classes approved:
% article   An article.  Standard printout.
% paper     A paper.  Used for more advanced, non-printed documents.
% proc      For demi-legal documents.
% book      For things that will never happen.
% letter    A letter.
% report    A report.  For rare cases.
\documentclass[a4paper,english]{book}
\usepackage{fixltx2e}
\usepackage{fontspec}
\usepackage{polyglossia}
\usepackage[osf]{mathpazo}
\setmainfont[Ligatures=TeX]{TeX Gyre Pagella}
\setsansfont[Ligatures={NoRequired,NoCommon,NoContextual}]{Myriad Pro} % ligatures in sans-serifs are ugly and unnecessary.
\setmonofont[]{DejaVu Sans Mono} % …and in monospace fonts, they are dumb (but DVSM has none no matter what)
\newfontfamily{\osn}[Numbers=OldStyle]{TeX Gyre Pagella}
\newfontfamily{\helv}[Numbers=Lining]{TeX Gyre Heros}
\newcommand{\helvetica}[1]{{\helv #1}}
\setcounter{secnumdepth}{5}
\setcounter{tocdepth}{5}
\usepackage{color}
\usepackage[usenames,dvipsnames,svgnames,table]{xcolor}
\usepackage{enumerate}
\usepackage{varioref}
\usepackage{setspace}
\PassOptionsToPackage{normalem}{ulem}
\usepackage{ulem}
\usepackage{hyphenat}
\usepackage[parfill]{parskip}
\setlength{\parskip}{\smallskipamount}
\setlength{\parindent}{0pt}
\usepackage{setspace}
\usepackage{marginnote}
\usepackage{xfrac}
\usepackage{graphicx}
\usepackage{booktabs}
\usepackage{array}
\usepackage{paralist}
\usepackage{verbatim}
\usepackage{subfig}
\usepackage{amsmath}
\usepackage{amssymb}
\usepackage{mathtools}
\usepackage{multirow}
\usepackage{tabularx}
\usepackage[unicode, colorlinks, breaklinks, pdftitle={PKGBUILDer Exceptions 2.0},pdfauthor={Warrick, Chris “Kwpolska”}]{hyperref}
\usepackage[vario]{fancyref}
\usepackage{upquote}
\usepackage{moreverb}
\def\hypdate{\kern.1em-\kern.1em}
\def\ndash{\ts--\hskip.25em}
\def\mdash{\ts---\hskip.25em}
\pdfpageheight\paperheight
\pdfpagewidth\paperwidth
\onehalfspacing

\newcommand{\noun}[1]{\textsc{#1}}
\providecommand{\tabularnewline}{\\}
\date{}
\makeatother
\numberwithin{equation}{section}
\usepackage[top=2cm, bottom=2cm, left=2cm, right=2cm]{geometry}
\nonfrenchspacing

\usepackage{titlesec}

\titleformat{\part}[display]
{\normalfont\Huge\filcenter}
{\fontsize{36pt}{1pc}\selectfont\textsc{Part} \thepart}
{0.5pt}
{\titlerule
\vspace{0.25pc}%
\fontsize{48pt}{1.2pt}\selectfont}

\titleformat{\chapter}[display]
{\normalfont\Large\filcenter}
{\textsc{\chaptertitlename}\Huge{} \thechapter}
{0.5pt}
{\titlerule
\vspace{0.25pc}%
\Huge}

\titleformat{\section}
{\normalfont\LARGE}
{\normalsize §\Huge\thesection}
{0.5em}{}

\titleformat{\subsection}
{\normalfont\Large}
{\footnotesize §\huge\thesubsection}
{0.5em}{}

\titleformat{\subsubsection}
{\normalfont\large}
{\scriptsize §\LARGE\thesubsubsection}
{0.5em}{}

\titleformat{\paragraph}[runin]
{\bf}
{¶\theparagraph}
{1em}{}

\titleformat{\subparagraph}[runin]
{\bf}
{\rm¶\bf\thesubparagraph}
{1em}{}

\titlespacing{\section}{0em}{0.1em}{0.1em}[0em]
\titlespacing{\subsection}{0em}{0.1em}{0.1em}[0em]
\titlespacing{\subsubsection}{0em}{0.1em}{0.1em}[0em]

\usepackage{fancyvrb}
\makeatletter
\def\PY@reset{\let\PY@it=\relax \let\PY@bf=\relax%
    \let\PY@ul=\relax \let\PY@tc=\relax%
    \let\PY@bc=\relax \let\PY@ff=\relax}
\def\PY@tok#1{\csname PY@tok@#1\endcsname}
\def\PY@toks#1+{\ifx\relax#1\empty\else%
    \PY@tok{#1}\expandafter\PY@toks\fi}
\def\PY@do#1{\PY@bc{\PY@tc{\PY@ul{%
    \PY@it{\PY@bf{\PY@ff{#1}}}}}}}
\def\PY#1#2{\PY@reset\PY@toks#1+\relax+\PY@do{#2}}

\expandafter\def\csname PY@tok@gd\endcsname{\def\PY@tc##1{\textcolor[rgb]{0.63,0.00,0.00}{##1}}}
\expandafter\def\csname PY@tok@gu\endcsname{\let\PY@bf=\textbf\def\PY@tc##1{\textcolor[rgb]{0.50,0.00,0.50}{##1}}}
\expandafter\def\csname PY@tok@gt\endcsname{\def\PY@tc##1{\textcolor[rgb]{0.00,0.27,0.87}{##1}}}
\expandafter\def\csname PY@tok@gs\endcsname{\let\PY@bf=\textbf}
\expandafter\def\csname PY@tok@gr\endcsname{\def\PY@tc##1{\textcolor[rgb]{1.00,0.00,0.00}{##1}}}
\expandafter\def\csname PY@tok@cm\endcsname{\let\PY@it=\textit\def\PY@tc##1{\textcolor[rgb]{0.38,0.63,0.69}{##1}}}
\expandafter\def\csname PY@tok@vg\endcsname{\def\PY@tc##1{\textcolor[rgb]{0.73,0.38,0.84}{##1}}}
\expandafter\def\csname PY@tok@m\endcsname{\def\PY@tc##1{\textcolor[rgb]{0.25,0.63,0.44}{##1}}}
\expandafter\def\csname PY@tok@mh\endcsname{\def\PY@tc##1{\textcolor[rgb]{0.25,0.63,0.44}{##1}}}
\expandafter\def\csname PY@tok@cs\endcsname{\def\PY@tc##1{\textcolor[rgb]{0.38,0.63,0.69}{##1}}\def\PY@bc##1{\setlength{\fboxsep}{0pt}\colorbox[rgb]{1.00,0.94,0.94}{\strut ##1}}}
\expandafter\def\csname PY@tok@ge\endcsname{\let\PY@it=\textit}
\expandafter\def\csname PY@tok@vc\endcsname{\def\PY@tc##1{\textcolor[rgb]{0.73,0.38,0.84}{##1}}}
\expandafter\def\csname PY@tok@il\endcsname{\def\PY@tc##1{\textcolor[rgb]{0.25,0.63,0.44}{##1}}}
\expandafter\def\csname PY@tok@go\endcsname{\def\PY@tc##1{\textcolor[rgb]{0.53,0.53,0.53}{##1}}}
\expandafter\def\csname PY@tok@cp\endcsname{\def\PY@tc##1{\textcolor[rgb]{0.00,0.44,0.13}{##1}}}
\expandafter\def\csname PY@tok@gi\endcsname{\def\PY@tc##1{\textcolor[rgb]{0.00,0.63,0.00}{##1}}}
\expandafter\def\csname PY@tok@gh\endcsname{\let\PY@bf=\textbf\def\PY@tc##1{\textcolor[rgb]{0.00,0.00,0.50}{##1}}}
\expandafter\def\csname PY@tok@ni\endcsname{\let\PY@bf=\textbf\def\PY@tc##1{\textcolor[rgb]{0.84,0.33,0.22}{##1}}}
\expandafter\def\csname PY@tok@nl\endcsname{\let\PY@bf=\textbf\def\PY@tc##1{\textcolor[rgb]{0.00,0.13,0.44}{##1}}}
\expandafter\def\csname PY@tok@nn\endcsname{\let\PY@bf=\textbf\def\PY@tc##1{\textcolor[rgb]{0.05,0.52,0.71}{##1}}}
\expandafter\def\csname PY@tok@no\endcsname{\def\PY@tc##1{\textcolor[rgb]{0.38,0.68,0.84}{##1}}}
\expandafter\def\csname PY@tok@na\endcsname{\def\PY@tc##1{\textcolor[rgb]{0.25,0.44,0.63}{##1}}}
\expandafter\def\csname PY@tok@nb\endcsname{\def\PY@tc##1{\textcolor[rgb]{0.00,0.44,0.13}{##1}}}
\expandafter\def\csname PY@tok@nc\endcsname{\let\PY@bf=\textbf\def\PY@tc##1{\textcolor[rgb]{0.05,0.52,0.71}{##1}}}
\expandafter\def\csname PY@tok@nd\endcsname{\let\PY@bf=\textbf\def\PY@tc##1{\textcolor[rgb]{0.33,0.33,0.33}{##1}}}
\expandafter\def\csname PY@tok@ne\endcsname{\def\PY@tc##1{\textcolor[rgb]{0.00,0.44,0.13}{##1}}}
\expandafter\def\csname PY@tok@nf\endcsname{\def\PY@tc##1{\textcolor[rgb]{0.02,0.16,0.49}{##1}}}
\expandafter\def\csname PY@tok@si\endcsname{\let\PY@it=\textit\def\PY@tc##1{\textcolor[rgb]{0.44,0.63,0.82}{##1}}}
\expandafter\def\csname PY@tok@s2\endcsname{\def\PY@tc##1{\textcolor[rgb]{0.25,0.44,0.63}{##1}}}
\expandafter\def\csname PY@tok@vi\endcsname{\def\PY@tc##1{\textcolor[rgb]{0.73,0.38,0.84}{##1}}}
\expandafter\def\csname PY@tok@nt\endcsname{\let\PY@bf=\textbf\def\PY@tc##1{\textcolor[rgb]{0.02,0.16,0.45}{##1}}}
\expandafter\def\csname PY@tok@nv\endcsname{\def\PY@tc##1{\textcolor[rgb]{0.73,0.38,0.84}{##1}}}
\expandafter\def\csname PY@tok@s1\endcsname{\def\PY@tc##1{\textcolor[rgb]{0.25,0.44,0.63}{##1}}}
\expandafter\def\csname PY@tok@gp\endcsname{\let\PY@bf=\textbf\def\PY@tc##1{\textcolor[rgb]{0.78,0.36,0.04}{##1}}}
\expandafter\def\csname PY@tok@sh\endcsname{\def\PY@tc##1{\textcolor[rgb]{0.25,0.44,0.63}{##1}}}
\expandafter\def\csname PY@tok@ow\endcsname{\let\PY@bf=\textbf\def\PY@tc##1{\textcolor[rgb]{0.00,0.44,0.13}{##1}}}
\expandafter\def\csname PY@tok@sx\endcsname{\def\PY@tc##1{\textcolor[rgb]{0.78,0.36,0.04}{##1}}}
\expandafter\def\csname PY@tok@bp\endcsname{\def\PY@tc##1{\textcolor[rgb]{0.00,0.44,0.13}{##1}}}
\expandafter\def\csname PY@tok@c1\endcsname{\let\PY@it=\textit\def\PY@tc##1{\textcolor[rgb]{0.38,0.63,0.69}{##1}}}
\expandafter\def\csname PY@tok@kc\endcsname{\let\PY@bf=\textbf\def\PY@tc##1{\textcolor[rgb]{0.00,0.44,0.13}{##1}}}
\expandafter\def\csname PY@tok@c\endcsname{\let\PY@it=\textit\def\PY@tc##1{\textcolor[rgb]{0.38,0.63,0.69}{##1}}}
\expandafter\def\csname PY@tok@ckw\endcsname{\let\PY@it=\textit\def\PY@tc##1{\textcolor[rgb]{0.0,0.66,0.86}{##1}}}
\expandafter\def\csname PY@tok@mf\endcsname{\def\PY@tc##1{\textcolor[rgb]{0.25,0.63,0.44}{##1}}}
\expandafter\def\csname PY@tok@err\endcsname{\def\PY@bc##1{\setlength{\fboxsep}{0pt}\fcolorbox[rgb]{1.00,0.00,0.00}{1,1,1}{\strut ##1}}}
\expandafter\def\csname PY@tok@kd\endcsname{\let\PY@bf=\textbf\def\PY@tc##1{\textcolor[rgb]{0.00,0.44,0.13}{##1}}}
\expandafter\def\csname PY@tok@ss\endcsname{\def\PY@tc##1{\textcolor[rgb]{0.32,0.47,0.09}{##1}}}
\expandafter\def\csname PY@tok@sr\endcsname{\def\PY@tc##1{\textcolor[rgb]{0.14,0.33,0.53}{##1}}}
\expandafter\def\csname PY@tok@mo\endcsname{\def\PY@tc##1{\textcolor[rgb]{0.25,0.63,0.44}{##1}}}
\expandafter\def\csname PY@tok@mi\endcsname{\def\PY@tc##1{\textcolor[rgb]{0.25,0.63,0.44}{##1}}}
\expandafter\def\csname PY@tok@kn\endcsname{\let\PY@bf=\textbf\def\PY@tc##1{\textcolor[rgb]{0.00,0.44,0.13}{##1}}}
\expandafter\def\csname PY@tok@o\endcsname{\def\PY@tc##1{\textcolor[rgb]{0.40,0.40,0.40}{##1}}}
\expandafter\def\csname PY@tok@kr\endcsname{\let\PY@bf=\textbf\def\PY@tc##1{\textcolor[rgb]{0.00,0.44,0.13}{##1}}}
\expandafter\def\csname PY@tok@s\endcsname{\def\PY@tc##1{\textcolor[rgb]{0.25,0.44,0.63}{##1}}}
\expandafter\def\csname PY@tok@kp\endcsname{\def\PY@tc##1{\textcolor[rgb]{0.00,0.44,0.13}{##1}}}
\expandafter\def\csname PY@tok@w\endcsname{\def\PY@tc##1{\textcolor[rgb]{0.73,0.73,0.73}{##1}}}
\expandafter\def\csname PY@tok@kt\endcsname{\def\PY@tc##1{\textcolor[rgb]{0.56,0.13,0.00}{##1}}}
\expandafter\def\csname PY@tok@sc\endcsname{\def\PY@tc##1{\textcolor[rgb]{0.25,0.44,0.63}{##1}}}
\expandafter\def\csname PY@tok@sb\endcsname{\def\PY@tc##1{\textcolor[rgb]{0.25,0.44,0.63}{##1}}}
\expandafter\def\csname PY@tok@k\endcsname{\let\PY@bf=\textbf\def\PY@tc##1{\textcolor[rgb]{0.00,0.44,0.13}{##1}}}
\expandafter\def\csname PY@tok@se\endcsname{\let\PY@bf=\textbf\def\PY@tc##1{\textcolor[rgb]{0.25,0.44,0.63}{##1}}}
\expandafter\def\csname PY@tok@sd\endcsname{\let\PY@it=\textit\def\PY@tc##1{\textcolor[rgb]{0.25,0.44,0.63}{##1}}}

\def\PYZbs{\char`\\}
\def\PYZus{\char`\_}
\def\PYZob{\char`\{}
\def\PYZcb{\char`\}}
\def\PYZca{\char`\^}
\def\PYZam{\char`\&}
\def\PYZlt{\char`\<}
\def\PYZgt{\char`\>}
\def\PYZsh{\char`\#}
\def\PYZpc{\char`\%}
\def\PYZdl{\char`\$}
\def\PYZhy{\char`\-}
\def\PYZsq{\char`\'}
\def\PYZdq{\char`\"}
\def\PYZti{\char`\~}
% for compatibility with earlier versions
\def\PYZat{@}
\def\PYZlb{[}
\def\PYZrb{]}
\makeatother

\usepackage{float}
\floatstyle{ruled}
\newfloat{listingf}{tbh}{lop}
\floatname{listingf}{Listing}

\newcommand{\pb}[0]{\textsc{pkgbuild}er}
\newcommand{\p}[1]{\nohyphens{\texttt{#1}}}
\begin{document}
\title{\pb{} \oldstylenums{3.0}}
\author{Chris “Kwpolska” Warrick}
\date{2013-03-19}


\begin{titlepage} \fontsize{48pt}{2pt}\selectfont \begin{center}
\null\vfil
\pb{} \osn 3.0

\Huge{} Chris \emph{Kwpolska} Warrick

\huge \rm 2013-03-19
\end{center}
\normalsize

~

The key words “MUST”, “MUST NOT”, “REQUIRED”, “SHALL”, “SHALL NOT”, “SHOULD”, “SHOULD NOT”, “RECOMMENDED”,  “MAY”, and “OPTIONAL” in this document are to be interpreted as described in RFC 2119.

Code listings were (more or less) heavily modified before inclusion.  Comments that are only in this document are marked with \PY{ckw}{\PYZsh{}} (as opposed to \PY{c}{\PYZsh{}}).

\emph{This document plans many improvements.  Thus, I decided to name the new version 3.0, because 2.2 doesn’t feel right given the scale of those improvements.}

\vfil\null

\end{titlepage}



\tableofcontents{}

\part{Exceptions}
\chapter{The current system}

It is ugly.  There is one exception: PBError.  It takes messages.  You know, \emph{text}.  To display for \emph{humans}, not \emph{machines}.  For example, like this:

\begin{listingf}
\begin{Verbatim}[commandchars=\\\{\}]
\PY{k}{raise} \PY{n}{PBError}\PY{p}{(}\PY{n}{\PYZus{}}\PY{p}{(}\PY{l+s}{\PYZsq{}}\PY{l+s}{AUR: HTTP Error \PYZob{}\PYZcb{}}\PY{l+s}{\PYZsq{}}\PY{p}{)}\PY{o}{.}\PY{n}{format}\PY{p}{(}\PY{n}{req}\PY{o}{.}\PY{n}{status\PYZus{}code}\PY{p}{)}\PY{p}{)}
\PY{k}{raise} \PY{n}{PBError}\PY{p}{(}\PY{n}{\PYZus{}}\PY{p}{(}\PY{l+s}{\PYZsq{}}\PY{l+s}{download: 0 bytes downloaded}\PY{l+s}{\PYZsq{}}\PY{p}{)}
\end{Verbatim}
\caption{Two sample exceptions raised in \pb{} \osn 2.1.6.3}

\end{listingf}

That’s uninformative.  What does the error code mean, exactly?  Not everybody has the \textsc{http} status codes memorized (and \emph{nobody} memorizes the more obscure ones, which shouldn’t appear in PKGBUILDer at all\footnote{\label{fn:httpcodes}Error codes that are likely to appear and be unhandled in \pb: \osn 403, 404, 500, 501, 503.  Status codes that are handled by the awesome \emph{Requests} library include 200, 301, 302.\rm}).  Also, what does the \texttt{AUR} part mean, exactly?  The place \emph{in the code} where this message was produced.  In our case, it is \p{pkgbuilder.aur.AUR().jsonreq()} and \p{pkgbuilder.build.Build().download()}.

\section{But wait, there’s more!}

The exceptions output are currently handled in \emph{three} places:
\begin{enumerate}[a)]
\item \p{main.main()} (see Listing \vref{mainl});
\item \p{build.Build().auto\_build()} (Listing \vref{buildl2}; this is the ugliest code in \pb);
\item \p{build.Build().build\_runner()} (Listing \vref{buildl1}).
\end{enumerate}

\begin{listingf}
\begin{Verbatim}[commandchars=\\\{\}]
\PY{k}{def} \PY{n+nf}{main}\PY{p}{(}\PY{n}{source}\PY{o}{=}\PY{l+s}{\PYZsq{}}\PY{l+s}{AUTO}\PY{l+s}{\PYZsq{}}\PY{p}{,} \PY{n}{quit}\PY{o}{=}\PY{n+nb+bp}{True}\PY{p}{)}\PY{p}{:}
    \PY{l+s+sd}{\PYZdq{}\PYZdq{}\PYZdq{}Main routine of PKGBUILDer.\PYZdq{}\PYZdq{}\PYZdq{}}
    \PY{k}{try}\PY{p}{:}
        \PY{ckw}{\PYZsh{} 200 (yes, exactly 200!) lines of logic}
    \PY{k}{except} \PY{n}{requests}\PY{o}{.}\PY{n}{exceptions}\PY{o}{.}\PY{n}{ConnectionError} \PY{k}{as} \PY{n}{inst}\PY{p}{:}
        \PY{n}{DS}\PY{o}{.}\PY{n}{fancy\PYZus{}error}\PY{p}{(}\PY{n+nb}{str}\PY{p}{(}\PY{n}{inst}\PY{p}{)}\PY{p}{)}
        \PY{c}{\PYZsh{} TRANSLATORS: do not translate the word \PYZsq{}requests\PYZsq{}.}
        \PY{n}{DS}\PY{o}{.}\PY{n}{fancy\PYZus{}error}\PY{p}{(}\PY{n}{\PYZus{}}\PY{p}{(}\PY{l+s}{\PYZsq{}}\PY{l+s}{PKGBUILDer (or the requests library) had }\PY{l+s}{\PYZsq{}}
                         \PY{l+s}{\PYZsq{}}\PY{l+s}{problems with fulfilling an HTTP request.}\PY{l+s}{\PYZsq{}}\PY{p}{)}\PY{p}{)}
        \PY{n+nb}{exit}\PY{p}{(}\PY{l+m+mi}{1}\PY{p}{)}
        \PY{ckw}{\PYZsh{} snip the exact same thing thrice, only with different exceptions}
    \PY{k}{except} \PY{n}{PBError} \PY{k}{as} \PY{n}{inst}\PY{p}{:}
        \PY{n}{DS}\PY{o}{.}\PY{n}{fancy\PYZus{}error}\PY{p}{(}\PY{n+nb}{str}\PY{p}{(}\PY{n}{inst}\PY{p}{)}\PY{p}{)}
        \PY{n+nb}{exit}\PY{p}{(}\PY{l+m+mi}{1}\PY{p}{)}

    \PY{n}{DS}\PY{o}{.}\PY{n}{log}\PY{o}{.}\PY{n}{info}\PY{p}{(}\PY{l+s}{\PYZsq{}}\PY{l+s}{Quitting.}\PY{l+s}{\PYZsq{}}\PY{p}{)}  \PY{ckw}{\PYZsh{} A very lonely line.}
\end{Verbatim}
\caption{\p{main.main()}}\label{mainl}
\end{listingf}

\begin{listingf}
\begin{Verbatim}[commandchars=\\\{\}]
\PY{k}{def} \PY{n+nf}{build\PYZus{}runner}\PY{p}{(}\PY{n+nb+bp}{self}\PY{p}{,} \PY{n}{pkgname}\PY{p}{,} \PY{n}{performdepcheck}\PY{o}{=}\PY{n+nb+bp}{True}\PY{p}{,}
                 \PY{n}{pkginstall}\PY{o}{=}\PY{n+nb+bp}{True}\PY{p}{)}\PY{p}{:}
    \PY{ckw}{\PYZsh{} docstring goes here}
    \PY{k}{try}\PY{p}{:}
        \PY{ckw}{\PYZsh{} snip 79 lines of logic}
            \PY{k}{if} \PY{n}{aurbuild} \PY{o}{!=} \PY{p}{[}\PY{p}{]}\PY{p}{:}
                \PY{k}{return} \PY{p}{[}\PY{l+m+mi}{72337}\PY{p}{,} \PY{n}{aurbuild}\PY{p}{]}
        \PY{ckw}{\PYZsh{} snip 43 lines}
    \PY{k}{except} \PY{n}{PBError} \PY{k}{as} \PY{n}{inst}\PY{p}{:}
        \PY{n}{DS}\PY{o}{.}\PY{n}{fancy\PYZus{}error}\PY{p}{(}\PY{n+nb}{str}\PY{p}{(}\PY{n}{inst}\PY{p}{)}\PY{p}{)}
        \PY{k}{return} \PY{p}{[}\PY{l+m+mi}{72789}\PY{p}{,} \PY{n+nb+bp}{None}\PY{p}{]}
    \PY{k}{except} \PY{n+ne}{IOError} \PY{k}{as} \PY{n}{inst}\PY{p}{:}
        \PY{n}{DS}\PY{o}{.}\PY{n}{fancy\PYZus{}error}\PY{p}{(}\PY{n+nb}{str}\PY{p}{(}\PY{n}{inst}\PY{p}{)}\PY{p}{)}
        \PY{k}{return} \PY{p}{[}\PY{l+m+mi}{72101}\PY{p}{,} \PY{n+nb+bp}{None}\PY{p}{]}
\end{Verbatim}

\caption{\p{build.Build().build\_runner().}}\label{buildl1}
\end{listingf}

\begin{listingf}
\begin{Verbatim}[commandchars=\\\{\}]
\PY{k}{def} \PY{n+nf}{auto\PYZus{}build}\PY{p}{(}\PY{n+nb+bp}{self}\PY{p}{,} \PY{n}{pkgname}\PY{p}{,} \PY{n}{performdepcheck}\PY{o}{=}\PY{n+nb+bp}{True}\PY{p}{,} \PY{n}{pkginstall}\PY{o}{=}\PY{n+nb+bp}{True}\PY{p}{)}\PY{p}{:}
    \PY{ckw}{\PYZsh{} docstring goes here}
    \PY{n}{build\PYZus{}result} \PY{o}{=} \PY{n+nb+bp}{self}\PY{o}{.}\PY{n}{build\PYZus{}runner}\PY{p}{(}\PY{n}{pkgname}\PY{p}{,} \PY{n}{performdepcheck}\PY{p}{,} \PY{n}{pkginstall}\PY{p}{)}
    \PY{n}{os}\PY{o}{.}\PY{n}{chdir}\PY{p}{(}\PY{l+s}{\PYZsq{}}\PY{l+s}{../}\PY{l+s}{\PYZsq{}}\PY{p}{)}
    \PY{k}{try}\PY{p}{:}
        \PY{k}{if} \PY{n}{build\PYZus{}result}\PY{p}{[}\PY{l+m+mi}{0}\PY{p}{]} \PY{o}{==} \PY{l+m+mi}{0}\PY{p}{:}
            \PY{n}{DS}\PY{o}{.}\PY{n}{fancy\PYZus{}msg}\PY{p}{(}\PY{n}{\PYZus{}}\PY{p}{(}\PY{l+s}{\PYZsq{}}\PY{l+s}{The build function reported a proper build.}\PY{l+s}{\PYZsq{}}\PY{p}{)}\PY{p}{)}
        \PY{k}{elif} \PY{n}{build\PYZus{}result}\PY{p}{[}\PY{l+m+mi}{0}\PY{p}{]} \PY{o}{\PYZgt{}}\PY{o}{=} \PY{l+m+mi}{0} \PY{o+ow}{and} \PY{n}{build\PYZus{}result}\PY{p}{[}\PY{l+m+mi}{0}\PY{p}{]} \PY{o}{\PYZlt{}} \PY{l+m+mi}{72000}\PY{p}{:}  \PY{c}{\PYZsh{} PBxxx.}
            \PY{k}{raise} \PY{n}{PBError}\PY{p}{(}\PY{n}{\PYZus{}}\PY{p}{(}\PY{l+s}{\PYZsq{}}\PY{l+s}{makepkg (or someone else) failed and }\PY{l+s}{\PYZsq{}}
                            \PY{l+s}{\PYZsq{}}\PY{l+s}{returned \PYZob{}\PYZcb{}.}\PY{l+s}{\PYZsq{}}\PY{p}{)}\PY{o}{.}\PY{n}{format}\PY{p}{(}\PY{n}{build\PYZus{}result}\PY{p}{[}\PY{l+m+mi}{0}\PY{p}{]}\PY{p}{)}\PY{p}{)}
            \PY{n+nb}{exit}\PY{p}{(}\PY{n}{build\PYZus{}result}\PY{p}{[}\PY{l+m+mi}{0}\PY{p}{]}\PY{p}{)}
        \PY{k}{elif} \PY{n}{build\PYZus{}result}\PY{p}{[}\PY{l+m+mi}{0}\PY{p}{]} \PY{o}{==} \PY{l+m+mi}{72789}\PY{p}{:}  \PY{c}{\PYZsh{} PBSUX.}
            \PY{k}{raise} \PY{n}{PBError}\PY{p}{(}\PY{n}{\PYZus{}}\PY{p}{(}\PY{l+s}{\PYZsq{}}\PY{l+s}{PKGBUILDer had a problem.}\PY{l+s}{\PYZsq{}}\PY{p}{)}\PY{p}{)}
            \PY{n+nb}{exit}\PY{p}{(}\PY{l+m+mi}{1}\PY{p}{)}
        \PY{k}{elif} \PY{n}{build\PYZus{}result}\PY{p}{[}\PY{l+m+mi}{0}\PY{p}{]} \PY{o}{==} \PY{l+m+mi}{72101}\PY{p}{:}  \PY{c}{\PYZsh{} I/O error.}
            \PY{k}{raise} \PY{n}{PBError}\PY{p}{(}\PY{n}{\PYZus{}}\PY{p}{(}\PY{l+s}{\PYZsq{}}\PY{l+s}{There was an input/output error.}\PY{l+s}{\PYZsq{}}\PY{p}{)}\PY{p}{)}
            \PY{n+nb}{exit}\PY{p}{(}\PY{l+m+mi}{1}\PY{p}{)}
        \PY{k}{elif} \PY{n}{build\PYZus{}result}\PY{p}{[}\PY{l+m+mi}{0}\PY{p}{]} \PY{o}{==} \PY{l+m+mi}{72337}\PY{p}{:}  \PY{c}{\PYZsh{} PBDEP.}
            \PY{ckw}{\PYZsh{} insert magic and recurrency here}

        \PY{k}{return} \PY{n}{build\PYZus{}result}
    \PY{k}{except} \PY{n}{PBError} \PY{k}{as} \PY{n}{inst}\PY{p}{:}
        \PY{n}{DS}\PY{o}{.}\PY{n}{fancy\PYZus{}error}\PY{p}{(}\PY{n+nb}{str}\PY{p}{(}\PY{n}{inst}\PY{p}{)}\PY{p}{)}
\end{Verbatim}
\caption{\p{build.Build().auto\_build()}, the ugliest code in \pb.}\label{buildl2}
\end{listingf}

\pagebreak
\section{Recap: why is it so \emph{evil}?}\label{sec:e:1:recap}
\osn The main problems are:

\begin{enumerate}
\item \emph{One} exception class that has only a \emph{human–only} (or even \emph{Chris–only!}) message;
\item Weird return codes (\p{build.Build().auto\_build()} and his friend \p{build.Build().build\_runner()});
\item Repetitiveness and general ugliness;
\item If anyone uses \pb{} as a library in his code (eg. aurqt, which hadn’t had any problems \emph{yet} and to which this part the document applies), they hate my \p{PBError}s. \rm
\end{enumerate}

\chapter{Exceptions \osn 2.0 \rm}


\section{How to fix it?}

\rm Well, just reverse the list in \fref{sec:e:1:recap} and get: \osn

\begin{enumerate}
\item One base class (which, in order to break backwards compatibility for various reasons, \emph{won’t be named} \p{PBError}), multiple subclasses with appropriate class members;
\item Replace the return codes with a true \p{\PY{k}{try}/\PY{k}{except}} block;
\item Make it look pretty and drop all the repeats.
\end{enumerate}

\rm Easy, wasn’t it?  Even better, it is not too hard to fix it.  It requires time and thinking.

\section{Proposed subclasses}

\osn Keep in mind that this isn’t the finished list, and it might be expanded.  Also, as a general rule, we hate people doing \p{from} imports and make them \p{import pkgbuilder.exceptions}.

\begin{enumerate}
\item Base class: \Large\bf PBException\normalsize\rm\osn
\begin{enumerate}

\item AURError (\p{\PY{n}{aur\PYZus{}response}\PY{o}{.}\PY{n}{results} \PY{k}{if} \PY{n}{aur\PYZus{}response}\PY{o}{.}\PY{n}{type} \PY{o}{==} \PY{l+s}{\PYZsq{}}\PY{l+s}{error}\PY{l+s}{\PYZsq{}}});
\item MakepkgError;
\item NetworkError;
\item PackageError (see Part II and \fref{chap:2:package}):
\begin{enumerate}
\item PackageNotFoundError;
\end{enumerate}
\item SanityError.
\end{enumerate}
\item IOError — handled in \p{main.main()} (now handled by the crazy \p{build.Build()} handlers!).
\end{enumerate}

\subsection{But where will those subclasses be?}

I plan to put them in a new module, named (obviously) \p{exceptions}.  It will contain all the exceptions listed above, and anyone who needs them will import them.  This practice is inspired by \emph{Requests} by Kenneth \nohyphens{Reitz}.

\chapter{Exception classes in depth}

\section{PBException}

This exception is used as a base exception.  All other exception inherit from this one.  It is also used for exceptions that don’t have their own classes yet.

\section{AURError}

This is the exception used for problems with the \textsc{aur rpc}.  When we get an answer, but it is an error, we should show it to the human verbatim, possibly throwing in some translations.

\section{MakepkgError}

Non–zero return codes of \p{makepkg}.  That is all.  Also, \emph{Resistance is futile.}  This means that we should not try to go anywhere further with this specific package when one occurs.  It should go up the stack to the first instance of \p{build.Build().auto\_build()}, preventing an infinite loop while building dependencies.

\section{NetworkError}

When anything in terms of the network breaks (didn’t get a response from the \textsc{aur rpc}, we throw a NetworkError.  Which should have a \p{origin} parameter, pointing to the \emph{Requests} exception.

\section{PackageError}

Anything that goes wrong regarding \p{Package}s.  The \p{Package} class is described in more detail in \fref{chap:2:package}.

\subsection{PackageNotFoundError}

No such package?  We throw this one instead.

\section{SanityError}

Any breakage and insanity goes here.  It is recommended to die as soon as a \p{SanityError} occurs.

\part{Improved \textsc{oop}}

\chapter{What and why}

Our current \textsc{oop} is bad.  I plan to create a \p{Package} class, containing the obvious things, in an even nicer format.  Bonus points for handling AUR output as \p{pkg.\_\_dict\_\_.update()} or a more human equvialent.  Certain other classes, on the contrary, don’t make sense.  This part describes the exact problems.

\chapter{The \p{Package} class and its subclasses} \label{chap:2:package}

The \p{Package} class will replace what is currently known as \p{pkg} in the \p{Build} functions.  The current \p{pkg} is a dict, coming directly from the \textsc{aur} or mimicking one as close as possible (\textsc{abs}), and in the future it will be an object of the \p{Package} class.  It will be compatible with both \textsc{aur} and \textsc{abs} packages, through two subclasses, \p{AURPackage} and \p{ABSPackage}.

\section{Planned properties}

Permitted types in italic.

\subsection{\p{Package}}
\begin{itemize}
\item \emph{str} name;
\item \emph{str} version;
\item \emph{str} description;
\item \emph{str} repo (Category for the \textsc{aur});
\item \emph{str} url;
\item \emph{str} licenses;
\item \emph{str} human (Maintainer/Packager).
\end{itemize}

\subsection{\p{AURPackage}}

\begin{itemize}
\item \emph{int} id;
\item \emph{bool} is\_outdated;
\item \emph{datetime.datetime (aware)} added;
\item \emph{datetime.datetime (aware)} modified;
\item \emph{int} votes;
\item \emph{str} urlpath.
\end{itemize}

\subsection{\p{ABSPackage}}

\begin{itemize}
\item \emph{str} architecture.
\end{itemize}

\chapter{Demolition of useless classes}

\emph{Useless} means half of them.  \p{Utils}, \p{Build} and \p{Upgrade} \textbf{go to hell}.  Having them as classes is unnecessary.  I will turn those into modules and functions.  Basically, I will throw out one level of indentation from the file and nuke all \p{\PY{n+nb+bp}{self}\PY{o}{.}} instances.  Some more fixes, moving some things and it should be fine.  Did I mention completely demolishing backwards compatibility?  Well, this is where it becomes visible very nicely.

\part{Force \p{--safeupgrade} for \pb}

\chapter{Rationale}

When pacman developers release a new version, you are asked to install it before any other upgrades.  \pb{} should do the same, but with one major change: using the \p{--safeupgrade} option, introduced in commit \p{0f91814e51} (merged in \p{72dda04c25}) and shipped with \rm 2.1.6.2. \osn

\section{How to pull it off?}

Steal the question from pacman localization files and ask it when we find a \pb{} upgrade.  When the user agrees, we need to run the \p{--safeupgrade} routine, which currently sits in \p{main.\textbf{main()}}.  It should be moved to \p{upgrade.pb\_failsafe()} or something like that.

\part{\p{cower -d} implementation}

\chapter{What does it do?}

That’s probably the easiest improvement: add an option to run \p{build.Build.build\_runner()}, stopping right before \p{os.chdir('./\{\}/'.format(pkg['Name']))}.  Also, we will split the first few package determination lines to another function while we are at it.

\section{Implementation}

There is one major problem: \p{-dDw} are already used.  We would need to find a better abbreviation.  \textbf{f}etch feels wrong IMO, because of \emph{force} which isn’t there \emph{yet} and won’t be in 3.0 (probably never, but still).  I could agree for a capital \textbf{F}, though.  And that is probably what will appear in the new version.

\subsection{Sample output}

\begin{listingf}
\begin{Verbatim}
[2/3] Downloading packages... [/] pkgbuilder
\end{Verbatim}

~

\hrule

~

\begin{Verbatim}
[3/3] Packages successfully downloaded
[|] Extracting...
\end{Verbatim}

~

\hrule

~

\begin{Verbatim}
[3/3] Packages successfully downloaded
[*] Packages successfully extracted
Downloaded: pkgbuilder-git pkgbuilder python2-nikola-git
\end{Verbatim}
\caption{Sample output of the -F command, in three stages.}
\end{listingf}

\part{Other improvements}

\chapter{Small fixes and improvements}
\begin{enumerate}
\item \p{\_\_all\_\_}.
\item One instance of pycman/pyalpm per \pb{} instance.
\item UI tools simillar to Tiedot.
\end{enumerate}

\part{Schedule}

\chapter{The schedule at a glance}

Names correspond to names of parts or chapters.  \(\Rightarrow\) is used to mark a dependency, i.e. \(\Rightarrow 1\) means that in order to do a stage, stage 1 needs to be completed.

Note that work is the most likely to happen during the weekend.

\begin{enumerate}[\bf {Stage} 1.\rm\osn]
\item Demolition of useless classes
\item Small fixes and improvements \(\Rightarrow 1\)
\item Force \p{--safeupgrade} for \pb \(\Rightarrow 1\)
\item \p{cower -d} implementation \(\Rightarrow 1, 2\)
\item Exceptions 2.0 \(\Rightarrow 1\)
\item The \p{Package} class and its subclasses \(\Rightarrow 1\)
\end{enumerate}

The full schedule will be available soon.

\end{document}